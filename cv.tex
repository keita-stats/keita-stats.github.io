\documentclass[11pt,a4paper]{article}
\usepackage[left=2cm,right=2cm,top=2cm,bottom=2cm]{geometry}
\usepackage{hyperref}
\usepackage{enumitem}

% No indentation for paragraphs
\setlength{\parindent}{0pt}
% No extra space between paragraphs
\setlength{\parskip}{0pt}

\begin{document}

\begin{center}
{\Large \textbf{Keita Nakamura}}\\[0.3em]
Department of Information Sciences, Graduate School of Science and Technology,\\
Tokyo University of Science\\
2641 Yamazaki, Noda-shi, Chiba 278-8510, Japan\\
\href{mailto:6324703@ed.tus.ac.jp}{6324703@ed.tus.ac.jp}
\end{center}

\section*{1. Education}

\noindent\textbf{Tokyo University of Science} (Chiba, Japan) \hfill \textit{Apr 2024 -- Present}\\
Doctoral Program in Information Sciences, Graduate School of Science and Technology.

\noindent\textbf{Yokohama City University} (Yokohama, Japan) \hfill \textit{Jun 2023 -- Mar 2024}\\
Research Student in Medical Science, Graduate School of Medicine.

\noindent\textbf{Tokyo University of Science} (Chiba, Japan) \hfill \textit{Apr 2021 -- Mar 2023}\\
Master of Science in Information Sciences, Graduate School of Science and Technology.

\noindent\textbf{Tokyo University of Science} (Chiba, Japan) \hfill \textit{Apr 2017 -- Mar 2021}\\
Bachelor of Science in Information Sciences, Faculty of Science and Technology.

\noindent\textbf{Tokyo Metropolitan Kunitachi Senior High School} (Tokyo, Japan) \hfill \textit{Graduated Mar 2017}

\section*{2. Teaching Experience}

\noindent\textbf{Teaching Assistant, Tokyo University of Science} \hfill \textit{Apr 2025 -- Present}\\
Probability Theory 1 and its Exercises, Mathematics for Information Sciences 2B and its Exercises

\noindent\textbf{Teaching Assistant, Tokyo University of Science} \hfill \textit{Oct 2024 -- Mar 2025}\\
Statistics 1 and its Exercises, Mathematics for Information Sciences 2A and its Exercises

\noindent\textbf{Teaching Assistant, Tokyo University of Science} \hfill \textit{Apr 2024 -- Sep 2024}\\
Probability Theory 1 and its Exercises, Mathematics for Information Sciences 2B and its Exercises

\noindent\textbf{Learning Support Center, Tokyo University of Science} \hfill \textit{Apr 2022 -- Mar 2023}\\
Academic support and tutoring for undergraduate students in mathematics.

\section*{3. Research}

\noindent\textbf{Tokyo University of Science} (Chiba, Japan) \hfill \textit{Apr 2024 -- Present}\\
\textit{Doctoral Research, under Prof. Kouji Tahata}\\
Graduate School of Science and Technology, Department of Information Sciences

\noindent\textbf{Yokohama City University} (Yokohama, Japan) \hfill \textit{Jun 2023 -- Mar 2024}\\
\textit{Research Student, Graduate School of Medicine}

\noindent\textbf{Tokyo University of Science} (Chiba, Japan) \hfill \textit{Apr 2021 -- Mar 2023}\\
\textit{Master's Thesis Research, under Prof. Kouji Tahata}\\
Graduate School of Science and Technology, Department of Information Sciences

\section*{4. Awards and Honors}

\noindent\textbf{Best Presentation Award, Japanese Society of Applied Statistics} \hfill \textit{May 2024}\\
Awarded for the presentation ``Symmetry and skew-symmetry of square contingency tables based on Aitchison geometry'' at the Japanese Society of Applied Statistics Annual Conference, Kyushu University.

\section*{5. Publications and Presentations}

\subsection*{5.1 Publications}

K. Nakamura, T. Nakagawa, K. Tahata: ``Quasi-Symmetry and Geometric Marginal Homogeneity: A Simplicial Approach to Square Contingency Tables'', \textit{Information Geometry}, 2025. DOI: \href{https://doi.org/10.1007/s41884-025-00176-1}{10.1007/s41884-025-00176-1}

K. Nakamura, T. Nakagawa, K. Tahata: ``Symmetry of Square Contingency Tables Using Simplicial Geometry'', \textit{Austrian Journal of Statistics}, Vol. 53, No. 4, pp. 85-98, 2024. DOI: \href{https://doi.org/10.17713/ajs.v53i4.1845}{10.17713/ajs.v53i4.1845}

\subsection*{5.2 Conference Presentations}

\subsubsection*{5.2.1 International Conferences}

[1] K. Nakamura, T. Nakagawa, K. Tahata: ``Geometric Marginal Homogeneity in Compositional Tables Based on Simplicial Geometry'', \textit{Further Developments of Information Geometry}, The University of Tokyo, Tokyo, Japan, 2025 (Poster session).

[2] K. Nakamura, T. Nakagawa, K. Tahata: ``Orthogonal decomposition of probability tables with Aitchison geometry for symmetry assessment'', \textit{CFE-CMStatistics}, King's College London, London, UK, 2024.

\subsubsection*{5.2.2 Domestic Conferences}

[1] K. Nakamura, T. Nakagawa, K. Tahata: ``On symmetry of contingency tables based on Aitchison geometry'', \textit{RIMS Joint Research (Group Type A) - Statistical Models and Their Effectiveness}, Research Institute for Mathematical Sciences, Kyoto University, March 2025.

[2] K. Nakamura, T. Nakagawa, K. Tahata: ``Quasi-symmetry analysis of cross-classified tables using Aitchison geometry'', \textit{Japanese Classification Society Annual Conference}, Nagoya Institute of Technology, November 2024.

[3] K. Nakamura, T. Nakagawa, K. Tahata: ``Symmetry and skew-symmetry of square contingency tables based on Aitchison geometry'', \textit{Japanese Society of Applied Statistics Annual Conference}, Kyushu University, May 2024. (\textbf{Best Presentation Award})

[4] K. Nakamura, T. Nakagawa, K. Tahata: ``Quasi-symmetry and geometric marginal homogeneity of square contingency tables using Aitchison geometry'', \textit{The Mathematical Society of Japan Annual Meeting}, Tohoku University, September 2023.

[5] K. Nakamura, T. Nakagawa, K. Tahata: ``Symmetry of square contingency tables using Aitchison geometry'', \textit{The Mathematical Society of Japan Annual Meeting}, Chuo University, March 2023.

\subsection*{5.3 Publications without Peer Review}

[1] K. Nakamura, T. Nakagawa, K. Tahata: ``On symmetry of contingency tables based on Aitchison geometry'', \textit{RIMS K\={o}ky\={u}roku}, Vol. 2318, pp. 1-12, 2025.

\section*{6. Research Grants and Funding}

[1] \textbf{Support for Pioneering Research Initiated by the Next Generation (SPRING)} by Japan Science and Technology Agency (JST) \hfill \textit{2024 -- Present}

\section*{7. References}

Prof. Kouji Tahata, Tokyo University of Science, \href{mailto:tahata@rs.tus.ac.jp}{tahata@rs.tus.ac.jp}

\end{document}
