\documentclass[11pt,a4paper]{article}
\usepackage[left=2cm,right=2cm,top=2cm,bottom=2cm]{geometry}
\usepackage{fontspec}
\setmainfont{Times New Roman}
\usepackage{xeCJK}
\setCJKmainfont{Hiragino Mincho ProN}
\usepackage{hyperref}
\usepackage{enumitem}

% No indentation for paragraphs
\setlength{\parindent}{0pt}
% Add spacing between paragraphs
\setlength{\parskip}{0.5em}
% Increase line spacing
\renewcommand{\baselinestretch}{1.1}

\begin{document}

\begin{center}
{\Large \textbf{Keita Nakamura}}\\[0.3em]
Department of Information Sciences, Graduate School of Science and Technology,\\
Tokyo University of Science\\
2641 Yamazaki, Noda-shi, Chiba 278-8510, Japan\\
\href{mailto:6324703@ed.tus.ac.jp}{6324703@ed.tus.ac.jp}
\end{center}

\section*{1. Education}

\noindent\textbf{Tokyo University of Science} (Chiba, Japan) \hfill \textit{Apr 2024 -- Present}\\
Doctoral Program in Information Sciences, Graduate School of Science and Technology.

\vspace{0.3em}
\noindent\textbf{Yokohama City University} (Yokohama, Japan) \hfill \textit{Jun 2023 -- Mar 2024}\\
Research Student in Medical Science, Graduate School of Medicine.

\vspace{0.3em}
\noindent\textbf{Tokyo University of Science} (Chiba, Japan) \hfill \textit{Apr 2021 -- Mar 2023}\\
Master of Science in Information Sciences, Graduate School of Science and Technology.

\vspace{0.3em}
\noindent\textbf{Tokyo University of Science} (Chiba, Japan) \hfill \textit{Apr 2017 -- Mar 2021}\\
Bachelor of Science in Information Sciences, Faculty of Science and Technology.

\vspace{0.3em}
\noindent\textbf{Tokyo Metropolitan Kunitachi Senior High School} (Tokyo, Japan) \hfill \textit{Graduated Mar 2017}

\section*{2. Teaching Experience}

\noindent\textbf{Teaching Assistant, Tokyo University of Science} \hfill \textit{Apr 2025 -- Present}\\
Probability Theory 1 and its Exercises, Mathematics for Information Sciences 2B and its Exercises

\vspace{0.3em}
\noindent\textbf{Teaching Assistant, Tokyo University of Science} \hfill \textit{Oct 2024 -- Mar 2025}\\
Statistics 1 and its Exercises, Mathematics for Information Sciences 2A and its Exercises

\vspace{0.3em}
\noindent\textbf{Teaching Assistant, Tokyo University of Science} \hfill \textit{Apr 2024 -- Sep 2024}\\
Probability Theory 1 and its Exercises, Mathematics for Information Sciences 2B and its Exercises

\vspace{0.3em}
\noindent\textbf{Learning Support Center, Tokyo University of Science} \hfill \textit{Apr 2022 -- Mar 2023}\\
Academic support and tutoring for undergraduate students in mathematics.

\section*{3. Research}

\noindent\textbf{Tokyo University of Science} (Chiba, Japan) \hfill \textit{Apr 2024 -- Present}\\
\textit{Doctoral Research, under Prof. Kouji Tahata}\\
Graduate School of Science and Technology, Department of Information Sciences

\vspace{0.3em}
\noindent\textbf{Yokohama City University} (Yokohama, Japan) \hfill \textit{Jun 2023 -- Mar 2024}\\
\textit{Research Student, Graduate School of Medicine}

\vspace{0.3em}
\noindent\textbf{Tokyo University of Science} (Chiba, Japan) \hfill \textit{Apr 2021 -- Mar 2023}\\
\textit{Master's Thesis Research, under Prof. Kouji Tahata}\\
Graduate School of Science and Technology, Department of Information Sciences

\section*{4. Awards and Honors}

\noindent\textbf{応用統計学会 優秀発表賞} \hfill \textit{2024年5月}\\
発表題目:「Aitchison幾何学に基づく正方分割表の対称性と歪対称性」, 応用統計学会, 九州大学.

\section*{5. Publications and Presentations}

\subsection*{5.1 Publications}

K. Nakamura, T. Nakagawa, K. Tahata: ``Quasi-Symmetry and Geometric Marginal Homogeneity: A Simplicial Approach to Square Contingency Tables'', \textit{Information Geometry}, 2025. DOI: \href{https://doi.org/10.1007/s41884-025-00176-1}{10.1007/s41884-025-00176-1}

\vspace{0.3em}
K. Nakamura, T. Nakagawa, K. Tahata: ``Symmetry of Square Contingency Tables Using Simplicial Geometry'', \textit{Austrian Journal of Statistics}, Vol. 53, No. 4, pp. 85-98, 2024. DOI: \href{https://doi.org/10.17713/ajs.v53i4.1845}{10.17713/ajs.v53i4.1845}

\subsection*{5.2 Conference Presentations}

\subsubsection*{5.2.1 International Conferences}

[1] K. Nakamura, T. Nakagawa, K. Tahata: ``Geometric Marginal Homogeneity in Compositional Tables Based on Simplicial Geometry'', \textit{Further Developments of Information Geometry}, The University of Tokyo, Tokyo, Japan, 2025 (Poster session).

\vspace{0.3em}
[2] K. Nakamura, T. Nakagawa, K. Tahata: ``Orthogonal decomposition of probability tables with Aitchison geometry for symmetry assessment'', \textit{CFE-CMStatistics}, King's College London, London, UK, 2024.

\subsubsection*{5.2.2 Domestic Conferences}

[1] 中村慶太, 中川智之, 田畑耕治: 「Aitchison幾何に基づく分割表の対称性について」, \textit{RIMS共同研究(グループ型A)統計モデルとその有効性}, 京都大学数理解析研究所, 2025年3月.

\vspace{0.3em}
[2] 中村慶太, 中川智之, 田畑耕治: 「Aitchison幾何による交差分類表の準対称性解析」, \textit{日本分類学会}, 名古屋工業大学, 2024年11月.

\vspace{0.3em}
[3] 中村慶太, 中川智之, 田畑耕治: 「Aitchison幾何学に基づく正方分割表の対称性と歪対称性」, \textit{応用統計学会}, 九州大学, 2024年5月. (\textbf{Best Presentation Award})

\vspace{0.3em}
[4] 中村慶太, 中川智之, 田畑耕治: 「Aitchison幾何を用いた正方分割表の準対称性と幾何周辺同等性」, \textit{日本数学会}, 東北大学, 2023年9月.

\vspace{0.3em}
[5] 中村慶太, 中川智之, 田畑耕治: 「Aitchison幾何を用いた正方分割表の対称性」, \textit{日本数学会}, 中央大学, 2023年3月.

\subsection*{5.3 Publications without Peer Review}

[1] 中村慶太, 中川智之, 田畑耕治: 「Aitchison幾何に基づく分割表の対称性について」, \textit{RIMS講究録}, Vol. 2318, pp. 1-12, 2025. \href{https://www.kurims.kyoto-u.ac.jp/~kyodo/kokyuroku/contents/pdf/2318-01.pdf}{[PDF]}

\section*{6. Research Grants and Funding}

[1] \textbf{JST次世代研究者挑戦的研究プログラム(SPRING)} \hfill \textit{2024年 -- 現在}\\
東京理科大学「イノベーティブ博士人材育成のための共創力強化プロジェクト」

\section*{7. References}

Prof. Kouji Tahata, Tokyo University of Science, \href{mailto:tahata@rs.tus.ac.jp}{tahata@rs.tus.ac.jp}

\end{document}
